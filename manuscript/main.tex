\documentclass[fleqn,10pt]{wlpeerj}
\title{Designing surveys for unmarked animal populations when estimation is model-based}

\author[1]{Paul B. Conn}
\author[1]{Erin E. Moreland}
\author[1]{Michael C. Cameron}
\author[1]{Erin Richmond}
\author[1]{Peter L. Boveng}

\affil[1]{National Marine Mammal Laboratory, NOAA-NMFS, Alaska Fisheries Science Center, 7600 Sand Point Way NE, Seattle, WA 98115 USA}

\keywords{animal abundance, bearded seal, line transect survey, model-based estimation, ringed seal, spatial regression, species distribution models, survey design}

\begin{abstract}
1.  Count and occupancy surveys are often used to estimate the density, abundance, and distribution of animal populations. Recently, model-based approaches to analyzing survey data have become popular because one can more readily accommodate departures from pre-planned survey routes and construct more detailed maps than one can with design-based procedures. Model-based analysis often makes use of species-covariate relationships and/or spatially autocorrelated random effects to help predict density or occurrence in unsampled locations.

2.  There is little information currently available to guide ecologists on appropriate survey design protocols for implementing efficient surveys when estimation is model-based. 

3. We investigate the interplay between the design of animal surveys and estimator performance (bias, precision) when statistical models are used to predict animal abundance and occurrence. In particular, we use simulation and a case study (designing an aerial survey for seals in the Chukchi Sea) to illustrate the relationship between design and estimator performance.

4. We show that disproportionately  targeting areas of high animal density or occupancy (preferential sampling) can lead to positive bias unless  there are covariates available to help control for variation in sampling intensity.  We also show that performance of a survey design can differ depending on the underlying estimation model.   Models which rely heavily on spatially autocorrelated random effects (e.g. geostatistical models) often perform well when sampling is spatially balanced.  For models that rely heavily on predictive covariates, investigators should disproportionately sample locations with high predictive variance.  For count studies, this will often mean greater sampling in locations where covariates suggest high abundance; for occupancy (presence-absence) surveys, this means greater effort in locations predicted to have occupancy near 0.5. 

5. Incorporating some level of spatial balance in survey designs can help minimize integrated prediction error in spatial models and can help produce estimators that are model-robust. Pilot data and/or two-stage sampling can help investigators identify and target additional locations with high prediction variance to further enhance estimator performance.  


\end{abstract}

\begin{document}

\flushbottom
\maketitle
\thispagestyle{empty}

\section*{Introduction}

Surveys of unmarked animal populations are often used to estimate abundance and occurrence of animal populations and to predict species distributions, enterprises central to conservation, ecology, and management. For studies of abundance, researchers have historically relied on design-based statistical inference \citep[e.g.][]{Cochran1977}, which requires adoption of a pre-defined sampling frame (e.g. using systematic random sampling, stratified random sampling, or some variant thereof).  Designing animal surveys is relatively straightforward in such applications, and unbiased point and variance estimators are available.  Recently, however, there has been a surge in research describing model-based procedures for estimating abundance, density, and occupancy from surveys of unmarked animals, including N-mixture models for repeated point counts \citep{Royle2004a}, occupancy models for presence-absence surveys \citep{MacKenzie2002,JohnsonEtAl2013}, and various model-based formulations for distance-sampling data \citep{HedleyBuckland2004,MillerEtAl2013,JohnsonEtAl2010}.  In such applications, it is common to use habitat or environmental covariates together with spatial effects (e.g. via trend surfaces or spatial random effects) to predict density or distributions across the landscape.  

Given that model-based inference is now routinely employed in ecology, it is perhaps surprising that there has been relatively little research directing ecologists how to allocate survey effort across space to improve predictive performance.  Notable exceptions include \citet{PeelEtAl2013}, who applied generalized additive models \citep[GAMs;][]{HastieTibshirani1999,Wood2006} to existing trawl survey data to compare alternative designs for reducing the variance of fish relative abundance indices, and \citet{FosterEtAl2014}, who considered alternative designs for autonomous underwater vehicle habitat mapping surveys.  These studies emphasize the importance of spreading sampling out over space (spatial balance) while maintaining a representative range of predictive covariates.
In the context of occupancy studies, \citet{PacificiEtAl2012} described a two-stage sampling design for increasing detection of rare species, showing that increased targeting of locations with high predicted occurrence  leads to occupancy estimators with better precision than simple random sampling.  However, this strategy did not perform as favorably when species were more abundant.


Although applications in ecology are scarce, there has been considerable research on optimal model-based spatial survey design in the statistical literature, for instance, to establish efficient pollution or air quality monitoring networks \citep[e.g.][]{CaseltonZidek1984,Muller2001}. 
Several options exist for optimizing designs, depending upon the type of model employed and whether one is interested in minimizing prediction variance across a landscape or minimizing uncertainty about parameter estimates \citep{Muller2001,DiggleLophaven2006,XiaEtAl2006}. For Gaussian, geostatistical applications, optimal (or near-optimal) designs often appear quite regular, perhaps with additional sample locations chosen close together to help estimate parameters describing spatial autocorrelation \citep{Lark2002,DiggleLophaven2006}.  However, as \citet{EvangelouZhu2012} argue, these criterion for optimality do not necessarily apply to spatial generalized linear mixed models (SGLMMs).  For instance, if animal abundance or occurrence is strongly linked to the value environmental covariates, it may be inefficient to survey as frequently in areas where values of these covariates strongly suggest presence (predicted occupancy near 1.0) or absence (predicted occupancy near 0.0).

If the ultimate goal of animal surveys is to  predict abundance or occurrence across a landscape, the notion of minimizing integrated prediction variance is an appealing design crierion.  However, the optimality of a given survey design is ultimately conditional on the assumed model for the data. An optimal survey design for one model could potentially be quite poor for another model \citep{Muller2001}.  For instance, the optimal design for a given sampling scenario may change depending upon the predictive covariates or functional forms for these covariates (e.g. linear, quadratic) that are included in the model.  The type of spatial basis used to account for spatial autocorrelation may also affect the optimality of the design \citep[e.g. via placement of knots across a landscape,][]{FinleyEtAl2009,GelfandEtAl2012}.  Since ecologists often have considerable uncertainty about the ultimate effect of covariates on animal abundance and distribution, the responsible survey planner would do well to choose a survey design that achieves high precision and low bias across a range of statistical models (i.e. a design that is model-robust).  For this reason, we focus here on developing reasonable strategies for making use of covariate data in designing surveys rather than invoking the advanced optimization (e.g. simulated annealing, exchange algorithms) needed to produce near-optimal designs conditional on a given model.

In addition to choosing designs that result in low integrated prediction variance (high precision), investigators should demand study designs and estimation models that result in unbiased estimators for the state variable of interest.  In a recent paper, \citet{DiggleEtAl2010} examined bias that can arise when preferential sampling is employed but not controlled for with relevant covariates during estimation.  This situation can arise when surveys target areas known to have high abundance or occupancy, but do not include this a priori criterion for sample inclusion in estimation.  


In this article, we explore the problem of designing surveys for unmarked animal populations for the purpose of estimating abundance and occurrence under model-based statistical inference.  We start by describing a common currency for notation and basic model structures considered in this paper.  Next, we review preferential sampling bias and how to avoid it, illustrated with a SGLMM applied to simulated count data. Next, we conduct a detailed simulation study to examine application of alternative survey designs on the performance of spatial N-mixture and occupancy models.  Our primary focus here is to determine reasonable allocations between locations selected for spatial balance versus locations selected to reduce predictive variance under two-stage sampling (equivalently, pilot data could also be used in conjunction with one stage sampling).  Finally, we consider the problem of developing an aerial transect survey design for ringed and bearded seals in the Chukchi Sea. The design in this case is complicated by logistical factors such as a low number of airports (2) and effective range of the aircraft.  However, data from historical surveys in the area are available to help simulate the likely effects of different design choices on estimator performance.    

\section*{Models and notation}

We focus here exclusively on discrete space models for animal encounter data as these seem to be the dominant form used in design and analysis of animal population surveys.  According to this approach, an investigator who wishes to make inference to a large study area breaks the  landscape up into $\mathcal{S}$ survey units (label these $U_1, U_2, \hdots, U_\mathcal{S}$), of which $S$ are selected for sampling (call the set of sampled locations ${\bf S}$. Each survey unit is assigned a vector of covariates, ${\bf x}_i$, and an indicator $H_i$ that takes on the value 1.0 if survey unit $i$ is sampled, and is 0 otherwise.  Animal abundance or occupancy is then written as a stochastic realization of a probability mass function $f()$:
\begin{eqnarray}
  \label{eq:process}
  Z_i & \sim & f(g^{-1}(\beta_0 + {\bf x}\boldsymbol{\beta}+\eta_i+\epsilon_i)). 
\end{eqnarray}
Here, $Z_i$ denotes the state variable of interest (e.g., occupancy or abundance), $g()$ is a link function (e.g. probit or logit for occupancy, log for count data), $\beta_0$ is an intercept parameter, $\boldsymbol{\beta} = \{ \beta_1, \beta_2, \hdots, \beta_K \}$ is a vector of regression parameters, $\eta_i$ is a spatially autocorrelated random effect, and $\epsilon_i$ is Gaussian error.  For occupancy, $f()$ would typically be Bernoulli, while the Poisson or negative binomial are typically choices for analysis of count data; common forms for $\eta_i$ include geostatistical models \citep{Cressie1993,DiggleEtAl1998}, Gaussian Markov random fields \citep[e.g. conditionally autoregressive models;][]{RueHeld2005}, or low rank alternatives such as predictive process \citep{BanerjeeEtAl2008,LatimerEtAl2009} or restricted spatial regression models \citep{Reich2006,Hughes2013}.

The model for $Z_i$ describes variation in the process of interest and is often described as the ``process'' model.  However, it is usually impossible to observe the system perfectly even in locations where sampling occurs, so it is customary to include an observation model describing incomplete detection.  For occupancy studies, the response variable $Y_i = 1$ if the species of interest is detected an 0 otherwise, and is modeled with a Bernoulli distribution \citep{Royle2008}
\begin{eqnarray}
  Y_i & \sim & \text{Bernoulli}(Z_i p_i),  
\end{eqnarray}
where $p_i$ is possibly a function of survey and observer specific covariates. Replicate surveys of the same sampling unit provide the necessary information to estimate $p_i$.   For count surveys, a possible model is 
\begin{eqnarray}
  \label{eq:obs_pois}
  Y_i & \sim & \text{Poisson}(Z_i A_i p_i),  
\end{eqnarray}
where the $Y_i$ now represents the count of animals obtained while surveying unit $i$, $A_i$ denotes the proportion of sample unit $i$ that is surveyed, and $p_i$ gives detection probability.  Additional information will often be needed to estimate $p_i$ in this context, such as data from double observers, distance observations, or double sampling \citep[see e.g.][]{BucklandEtAl2001,Royle2004,Borchers2006,ConnEtAl2014}.

For the remainder of this treatment, we use bold symbols to denote vector-valued quantities or matrices.  We also use standard bracket notation to denote probability mass and density functions.  For instance $[{\bf Z}]$ denotes the marginal probability mass function for ${\bf Z}$, and  $[{\bf Z}|{\bf Y}]$ represents the conditional distribution of ${\bf Z}$ given ${\bf Y}$.

\section*{Preferential sampling}

One of the appealing aspects of model-based estimation is that there is no requirement that surveys rely on a pre-planned survey design selected probabilistically from an underlying sampling frame.  For instance, investigators can reallocate sampling effort if weather or logistics preclude surveying in a desired location.  This can be a crucial advantage in surveys covering large areas with frequent inclement weather.  

However, the manner in which effort is ultimately allocated can potentially have profound influence on estimator performance.  With respect to nonrandom sampling, two possible problems seem particularly likely: coarse scale preferential sampling (CSPS), and fine scale preferential sampling (FSCS) (Fig. \ref{fig:pref}).  FSPS arises when the observations taken at a  particular sampling unit are non-random with respect to the density of animals within that sampling unit.  For instance, when allocating line transect survey effort to a particular sampling unit, it may be tempting to place the transect in a manner that targets habitat or landscape features that maximize the number of animals that will be encountered.  Depending upon the interpretation of occupancy, this may or may not be reasonable.  However, if trying to estimate density or abundance, this strategy will clearly often lead to positive bias.

By contrast, CSPS arises when the locations being sampled and the process of interest (e.g. density, occupancy) are conditionally dependent given modeled covariates \citep{DiggleEtAl2010}.  For instance, CSPS can occur when the investigator uses a priori knowledge or observations of the state variable obtained during sampling to allocate survey effort in places where abundance or occurrence is known to be high. \citet{DiggleEtAl2010} showed that this type of preferential sampling can lead to bias when this extra information is not included in models for the state variable of interest.  Specifically, CSPS arises when we consider the set of sampled locations as stochastic and when $[{\bf S},\boldsymbol{\eta},\boldsymbol{\epsilon}] \neq [{\bf S}][\boldsymbol{\eta},\boldsymbol{\epsilon}]$ \citep{DiggleEtAl2010}.  Bias can then result because spatial regression models require that $E(\boldsymbol{\theta})=E(\boldsymbol{\epsilon})=0$ but cells are targeted specifically because density or occupancy is higher than average.  

Bias attributed to CSPS may seem counterintuitive, especially given the maxim in survey sampling to allocate more effort to strata for which animal density is high. For instance, in large scale line transect surveys under stratified sampling, the optimal amount of effort that should be allocated to stratum $s$ is $A_s D_s^{0.5}$, where $A_s$ is the area of $s$ and $D_s$ is the anticipated density \citep[][eqn 7.7]{BucklandEtAl2001}.  Thus, there are theoretical reasons to sample more in high density areas than in low density areas. The obvious solution in this instance is to compensate for CSPS in model-based inferences by accounting for variation in sampling intensity with explanatory covariates or post hoc stratification. However, this does not always work when effort is allocated in a subjective manner.

To illustrate CSPS, we conducted a small count survey simulation experiment.  For each of 100 simulations, we generated abundance of a hypothetical species over a $30 \times 30$ grid as a function of two spatially autocorrelated covariates (with log-linear effects of covariates and an interaction term; Appendix S1).   For each simulated landscape we generated count data using Eqns \ref{eq:process} \& \ref{eq:obs_pois}, setting $A_i = p_i = 1.0$ for demonstration purposes.  We conducted three types of surveys of $K=50$ survey units: (i) a spatially balanced survey, (ii) an unequal probability design where the probabilities of sample inclusion was a function of covariate values (see Appendix S1), and an unequal probability design where the probability of sample inclusion was set to $(Z_i+1)/\sum(Z_i+1)$ (representing CSPS).  We provided a count-based process model with the first of the covariates as an explanatory variable, treating the second as a variable conspiring to influence distribution and abundance and imparting spatial autocorrelation but which remains unknown to the investigator. We used Markov chain Monte Carlo to conduct statistical inference, generating posterior predictions as $N = \sum_i Z_i$ and compared the performance (bias, precision, 90\% credible interval coverage) of these predictions relative to known, true values.


However, the best guard against CSPS is to make decisions on sampling based on the values of explanatory covariates rather than 

Chakraborty et al. acknowledge it's an issue for SDMs but don't attempt to model it.

\section*{Example: simulated occupancy and repeated point count surveys}






\section*{Example: planning an aerial Arctic seal survey}

Ecologists at NOAA's National Marine Mammal Laboratory plan to conduct aerial surveys of ringed and bearded seals over the eastern Chukchi Sea in the spring of 2016.  A previous survey \citep{Bengtson2005} indicated that the density of both species was higher in southern portions of the Chukchi Sea and in areas closest to mainland Alaska. Thus, in addition to  



These surveys will complement earlier (2012, 2013) surveys of the Bering Sea, and will largely use the same sampling methodology.  In particular, thermal video cameras mounted through the belly port of the aircraft will provide initial detection of seals that are basking on sea ice, and high resolution digital cameras will provide information on species identity \citep{ConnEtAl2014}.  We anticipate using a model-based approach to estimation, following the hierarchical model structure outlined by \citet{ConnEtAl2014} for Bering Sea surveys.  This approach includes a spatial regression model for the distribution of seals among survey units, but the observation model is considerably complicated by imperfect detection (thermal detection misses $\approx 6\%$ of seals), incomplete availability (satellite telemetry records indicate that 40-70\% of seals are basking at the time of surveys), and incomplete coverage of the digital photograph records (20-25\% of thermally detected seals are not photographed).  \citet{ConnEtAl2014} also included species misclassification in their observation model, but updated estimates of species misclassification probabilities \citep{McClintockEtAlInReview} indicate that the probability of misclassifying a bearded seal as a ringed seal and vice versa is negligible.  However, there remains a small probability ($<2\%$) of seal species being recorded as ``unknown'', even if a thermal detection has an associated photograph.



\citet{Bengtson2005} estimated abundance in 12 spatial strata, using a combination of long and short flight tracks to sample more in areas of higher anticipated seal density.  With the goal of treating these estimates as pilot data, we have attempted to reproduce density maps from \citet{Bengtson2005}, in some cases extending strata with low density northward and eastward from their original study area to the U.S. Exclusive Economic Zone boundary (Fig. \ref{fig:Chukchi}). We have 




\section*{Discussion}


In the context of polynomial response surface design, \citet{BoxDraper1959} showed that when one wishes to reduce predictive variance, but also has uncertainty about the functional form of the response surface to be estimated, that the ``optimal" strategy in terms of bias and variance combined is often to concentrate on minimizing bias alone.  For this purpose, a first order orthogonal design is required; in spatial applications easting and northing covariates, this would imply some type of systematic sample.  ``. . . conclusions obtained by minimizing variance error only and assuming the postulated model to be correct are likely to be wrong in many practical design situations." (Draper and Smith 1966, pp 18-19).

For instance, one could augment a spatially balanced sampling design with a number of locations which are known or thought to have high leverage (as a function of available covariates, for instance).



A number of authors have explored optimal knot placement in spatial models.  In the context of predictive process modeling \citep[where a covariance function is specified over a group of knots; see][]{BanerjeeEtAl2008}, \citet{FinleyEtAl2009} and \citet{GelfandEtAl2013} considered near-optimal knot placement by minimizing spatially averaged prediction variance.
Gelfand et al. knot selection

Using GAs to select knot placement using


Contrast with posterior loss (e.g., Jay's linex loss function estimator)

Contrast with cross validation

David Miller (the evil one)/Simon Wood stuff on edge effects

Contrast with other approaches-
Gelfand et al. Bayesian analysis - intrinsic CAR in SDM
Chakraborty et al. '10 - spatial abundance modeling - ordinal
Latimer et al. 2009 - spatial predictive process modelling in SPDs

Can't be complacent... still possible to get poor/biased results, e.g. if $\tau_\epsilon \rightarrow 0$.  Can't resolve pathological problems.

Presence-absence data, other link functions (e.g. probit)

extensions to models w/ secondary observation process, measurement error

much attention has been given to collinearity in multiple linear regression - suggest researchers give as much attention to predictive extrapolation bias in predictive models

For predictions with spatial , our experience is that predictions outside the minimum convex polygon where data are obtained can sometimes be more problematic than predictions within the polygon.  Spatial prediction surfaces may have a tendency to bend up or down in these areas, resulting in ``edge effects" that can lead to positive prediction bias when a log link function is employed \citep{VerHoefJansen2007}.






Judicious choices of priors can at times limit the influence of priors on calculation of the gIVH.  For example, if we specify a Zellner's g-prior \citep{Zellner1986} on the vector of regression coefficients, $[\boldsymbol{\beta}] = \textrm{MVN}({\bf 0},(\tau_\beta X^\prime X)^{-1})$, then priors are automatically scaled to the level of the gathered covariates, and the only remaining task is to investigate sensitivity to $\tau_\beta$.
Some care must be taken with prior distributions if Eq. \ref{eq:E_var_lambda} is used for calculation of prediction variance.  For instance, standard ``non-informative" or ``flat" priors may place substantial mass on implausible values.  When using the gIVH or prediction variance to help guide sampling design \citep[see][for an example using prediction variance]{DiggleLophaven2006}, we suggest using informative prior distributions.

Thus, it may prove useful when formulating survey designs since one can examine whether or not prediction points lie within the gIVH without ever collecting response data there.



Results from automated simulation routines must be taken within an appropriate context.  In real life situations, researchers often are privy to a wealth of ecological intuition and knowledge, and would likely disbelieve implausibly high or low estimates of abundance in certain areas and to adjust estimation models accordingly.  Nevertheless, it is important to



\subsection*{Figures and Tables}

Use the table and tabular commands for basic tables --- see Table~\ref{tab:widgets}, for example. You can upload a figure (JPEG, PNG or PDF) using the project menu. To include it in your document, use the includegraphics command as in the code for Figure~\ref{fig:view} below.

\begin{figure}[ht]
\centering
\caption{A depiction of two types of preferential sampling.  In (A), an investigator preferentially places point transects (red squares) within regions of high known density (blue polygons).  This can cause bias in abundance or occupancy estimators unless this a priori knowledge about density is explicitly modeled.  In (B), a fine scale version of preferential sampling occurs when a line transect (red line) is intentionally placed across a region of high quality habitat.  If a landscape is discretized into homogeneous survey units (as in a grid), it is essential that the habitat surveyed within each survey unit be randomly determined.  If not, bias (usually positive) can be expected.}
\includegraphics[width=\linewidth]{Pref_sampling}
\label{fig:pref}
\end{figure}

\begin{figure}[ht]
\centering
\caption{Planned study area for 2016 arctic seal surveys in the eastern Chukchi Sea off of Alaska. The study area is discretized into $625km^2$ grid cells on a Polar Stareophonic projection (gray lines), commensurate with the resolution of available sea ice covariate data.  The boundaries of the study area are determined by the Bering Strait to the south, the U.S. Exclusive Economic Zone to the west and north (yellow and black dashed line), and the $156^{\circ}W$ longitude line to the west (red line).  Also shown are the 10 near shore (yellow shading) and 2 off shore (green shading) employed in previous seal surveys in this area \citep{Bengtson2005}.  Water depths >500m are indicated in purple.}
\includegraphics[width=\linewidth]{LOSI_Planning_2016}
\label{fig:Chukchi}
\end{figure}

\begin{figure}[ht]
\centering
\caption{A collection of 9 hypothetical aerial transect designs for seals over the eastern Chukchi Sea, differing by (i) number of flights (4, 8, or 12; displayed on columns.), and (ii) the spatial distribution of tracks.  The first row includes scenarios where only ``long" tracks are flown, trying to achieve more or less even coverage across the study area, while subsequent rows have an increasingly greater number of short tracks flown near the coast to target areas of higher seal densities.}
\includegraphics[width=\linewidth]{sim_flights_Chukchi}
\label{fig:flights}
\end{figure}

%\begin{table}[ht]
%\centering
%%\begin{tabular}{l|r}
%Item & Quantity \\\hline
%Widgets & 42 \\
%Gadgets & 13
%\end{tabular}
%\caption{\label{tab:widgets}An example %table.}
%\end{table}

\begin{table}[ht]
\label{tab:PS}
\caption{Performance of count-based abundance estimators as a function of survey design and estimation model.  Performance measures include proportional relative bias (``Bias"), root mean squared error (``RMSE"), coefficient of variation (``CV"), and 90\% credible interval coverage (``Cov90''; the proportion of simulations where true abundance was between the 5th and 95th percentiles of posterior samples).  Mean values over 400 simulation replicates are presented for spatially balanced sampling (``Balanced"), inverse probability sampling based on covariate values (``Covariate"), and inverse probability sampling based on a priori knowledge of areas of high abundance (``Preferential"). In addition, two different estimation models were applied to each dataset, including a generalized linear model (``GLM''), and a spatial regression model (``RSR'').} 
\centering
\begin{tabular}{rrlrrrr}
  \hline
 Design & Model & Bias & RMSE & CV & Cov90 \\ 
  \hline
Balanced & GLM & 0.00 & 1038 & 0.069 & 0.92 \\ 
Balanced & RSR & 0.00 & 1034 & 0.057 & 0.90 \\ 
Covariate & GLM & 0.00 & 1392 & 0.067 & 0.88 \\ 
Covariate & RSR & 0.00 & 1152 & 0.061 & 0.88 \\ 
Preferential & GLM & 0.21 & 5807 & 0.060 & 0.24 \\ 
Preferential & RSR & 0.15 & 3040 & 0.054 & 0.29 \\ 
   \hline
\end{tabular}
\end{table}


\section*{Results and Discussion}


\section*{Acknowledgments}

So long and thanks for all the fish.

\bibliography{master_bib}

\end{document}